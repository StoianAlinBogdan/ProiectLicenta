\documentclass[12pt]{report}

%% limba document
\usepackage[romanian]{babel}
\selectlanguage{romanian}

%% font lucrare
\usepackage[utf8]{inputenc}
\usepackage[T1]{fontenc}
\usepackage{mathptmx}

%% geometria paginii
\usepackage{geometry}
%% setari antet/subsol
\usepackage{fancyhdr}
%% indentare primul paragraf din capitol
\usepackage{indentfirst}
\usepackage[unicode, hidelinks, colorlinks, linkcolor=black, urlcolor=blue, citecolor=black]{hyperref}
%% culoare text (pe moment folosit in titluri de capitole)
%% https://www.overleaf.com/learn/latex/Using_colours_in_LaTeX
\usepackage[dvipsnames]{xcolor}
%% format titlu capitol/subcapitol/etc
\usepackage{titlesec}
%% titluri pentru figuri/tabele/listing-uri de cod/etc.
\usepackage{caption}
%% pozitionare float-uri (figuri/tabele/etc)
\usepackage{float}
%% pentru tabele pe mai multe pagini
\usepackage{longtable}	
%% pentru tabele cu celule cu mai multe linii
\usepackage{multirow}			
%% pentru ecuatii
\usepackage{amsmath}
%% pentru definitii, teoreme
\usepackage{amsthm}
%% pentru algoritmi, pachetul algorithm este cel mai basic
\usepackage[chapter]{algorithm}
\usepackage[noend]{algpseudocode}
% daca pachetul algorithm nu este satisfacator pentru ceea ce aveti de scris, mai exista pachetele:
% algorithmic, algorithmicx, algpseudocode, algorithm2e
% https://tex.stackexchange.com/questions/229355/algorithm-algorithmic-algorithmicx-algorithm2e-algpseudocode-confused
%% formatari listing
\usepackage{listings}
% https://www.overleaf.com/learn/latex/Code_listing
%% pachetul minted formateaza/coloreaza sintaxa codului sursa, spre deosebire de pachetul listings care doar face bold pe sintaxa
\usepackage[newfloat]{minted}
% https://www.overleaf.com/learn/latex/Code_Highlighting_with_minted


%% setup dimensiune pagina, dimensiune header/footer
%% setup dimensiune 16pt
\input{macros/dimensiuni}
%% date caracteristice universitatii/facultatii.
%% in acest fisier nu trebuie aduse modificari, cu exceptia tipului de teza si a anului in care este sustinuta teza
%% comune: coperta externa si interna
\newcommand{\university}    {UNIVERSITATEA TEHNICĂ „Gheorghe Asachi” din IAȘI}
\newcommand{\faculty}       {FACULTATEA DE AUTOMATICĂ ȘI CALCULATOARE}
\newcommand{\studyfieldlbl} {DOMENIUL: }
\newcommand{\studyfield}    {Calculatoare și Tehnologia Informației}
\newcommand{\studyproglbl}  {SPECIALIZAREA: }
\newcommand{\studyprog}     {Tehnologia Informației}
\newcommand{\promotion}     {2022}
\newcommand{\location}      {Iași}

%% absolventii de master vor inlocui "diploma" cu "disertatie"
\newcommand{\thesistype}    {Lucrare de diplomă}
%\newcommand{\thesistype}    {Lucrare de disertație}
%% datele specifice: nume, prenume student; titlu lucrare; nume, prenume, grad didactic coordonator
%% PENTRU MODIFICARE SE EDITEAZA FISIERUL macros/student.tex
%% date student si lucrare
\newcommand{\thesistitle}   {Analiză comparativă a metodelor clasice și cuantice
de generare a numerelor aleatorii. Aplicație.}    %<---------
\newcommand{\authorlast}    {Stoian}         %<---------
\newcommand{\authorfirst}   {Alin-Bogdan}
\newcommand{\authornamefl}  {\authorfirst \space \authorlast} % first name last
\newcommand{\authornamelf}  {\authorlast \space \authorfirst} % last name first

%% titlul academic si numele coordonatorului stiintific
\newcommand{\coordinator}   {Ș.l. dr. ing. Petrilă Iosif-Iulian}
%% pentru numele complet si grad didactic/titlu academic, consultati https://ac.tuiasi.ro/despre/personal/cadre-didactice-ale-departamentului-de-calculatoare/



\begin{document}

%% definitii culori
\definecolor{albastru-ac}{HTML}{004586}

\titleformat
    {\chapter}
    [hang]
    {\color{albastru-ac}\bfseries\large}
    {}
    {0pt}
    {}
    []
    
\titlespacing
    {\chapter}%
    {1.27cm}
    {0.42cm}
    {0.21cm}
    [0pt]

\titleformat
    {\section}
    [hang]
    {\color{albastru-ac}\bfseries\large}
    {}
    {0pt}
    {}
    []
    
\titlespacing
    {\section}%
    {1.27cm}
    {0.42cm}
    {0.21cm}
    [0pt]

\begin{center}
    \large
    {\textbf{\thesistype} -- \textbf{Raport nr. 1}}
    
    \vspace{0.5cm}
    
    \normalsize
    \textbf{\thesistitle}
\end{center}

\textit{Student}: \textbf{\authornamefl}

\textit{Coordonator științific}: \textbf{\coordinator}

\vspace{0.5cm}

Primul raport trebuie să aibă minim 2-3 pagini și să conțină:

\section*{Scopul temei alese}

În acest capitol trebuie evidențiat domeniul mai larg în care se încadrează lucrarea (de exemplu: motoare de căutare web, inteligență artificială, securitate etc.), eventual subdomeniul. Se prezintă, pe scurt problemele specifice domeniului, respectiv subdomeniului și ce problema/probleme încearcă să rezolve proiectul propus.

\section*{Referințe la temă/subiecte similare}

În urma documentării făcute în primele luni de după alegerea temei, vă rezultă o serie de subiecte/soluții/aplicații similare dezvoltate de companii sau persoane individuale, pe care le puteți încadra în domeniul/subdomeniul în care v-ați propus să realizați proiectul. 

Acest capitol este dedicat prezentării acestor subiecte/soluții/aplicații similare, în parte, în care veți:
\begin{itemize}
    \item referi resursa online/numele dezvoltatorului;
    \item descrie în câteva rânduri produsul/soluția;
    \item face o analiză critică a ceea ce considerați că sunt puncte tari/puncte slabe ale subiectului similar -- \textit{aici intervine originalitatea rezultată prin trecerea prin filtrul gândirii voatre};
    \item putea indica ce funcționalități existente deja în acea soluție vreți și voi să le implementați mai bine;
    \item putea indica ce funcționalități lipsă ați descoperit, le considerați necesare (argumentând aceasta) și probabil că vreți să le implementați în proiectul lucrării de licență.
\end{itemize}

Un avantaj al acestui proiect este că puteți utiliza aceeași resursă bibliografică pe care o folosiți ca bază pentru lucrarea de licență. De exemplu, detalii asupra protocolului HTTP puteți consulta în \cite{misc:web:rfc7231}.

La finalul acestui capitol este necesară o concluzie privind stadiul actual al soluțiilor similare, eventual comparativ.

\section*{Resurse hardware/software utilizate}

Aici nu începeți să enumerați specificațiile tehnice ale laptopului vostru!

Rolul acestui capitol este de a indica:
\begin{itemize}
    \item anumite constrângeri hardware necesare pentru soluția în curs de dezvoltare la momentul redactării acestui raport. De exemplu, dacă folosiți mai multe mașini virtuale, va fi necesară prezentarea resurselor prevăzute pentru acestea (nr. core CPU, RAM, stocare etc.);
    \item sistemul/sistemele de operare, limbajul/limbajele de programare necesare pentru dezvoltarea proiectului, argumentarea alegerii lor, eventual mediul de programare și o minimă motivație privind avantajele folosirii acestuia.
\end{itemize}

În situația în care dezvoltați proiectul pe un dispozitiv și îl testați pe alte dispozitive, acestea pot fi descrise.

Dacă nu sunt necesare cerințe speciale hardware, ci unele comune, se specifică acest lucru.

\section*{Algoritmi sau metode alese}

În cadrul acestei secțiuni pot fi descriși o parte dintre algoritmii studiați și/sau metodele de soluționare ale problemei țintă. Nu trebuie să includă detalii foarte exacte întrucât este foarte posibil ca o parte dintre aceste chestiuni să se schimbe până în final. Totuși, în linii mari, trebuie să descrieți modul de lucru.

\section*{Rezultate așteptate}

Secțiunea ar trebui să includă o descriere a țintei finale a proiectului de diplomă/disertație (după caz). Ce vă propuneți să rezolvați pentru problema aleasă? Puteți contura rezultatele practice pe care vă așteptați/vă propuneți să le obțineți în final sau puteți evidenția un studiu teoretic al unor algoritmi sau concepte.

% *************** referinte bibliografice ***************
\nopagebreak
\bibliographystyle{IEEEtran}
\bibliography{bibliografie/bibliografie}

\end{document}