\documentclass[12pt]{report}

%% limba document
\usepackage[romanian]{babel}
\selectlanguage{romanian}

%% font lucrare
\usepackage[utf8]{inputenc}
\usepackage[T1]{fontenc}
\usepackage{mathptmx}

%% geometria paginii
\usepackage{geometry}
%% setari antet/subsol
\usepackage{fancyhdr}
%% indentare primul paragraf din capitol
\usepackage{indentfirst}
\usepackage[unicode, hidelinks, colorlinks, linkcolor=black, urlcolor=blue, citecolor=black]{hyperref}
%% culoare text (pe moment folosit in titluri de capitole)
%% https://www.overleaf.com/learn/latex/Using_colours_in_LaTeX
\usepackage[dvipsnames]{xcolor}
%% format titlu capitol/subcapitol/etc
\usepackage{titlesec}
%% titluri pentru figuri/tabele/listing-uri de cod/etc.
\usepackage{caption}
%% pozitionare float-uri (figuri/tabele/etc)
\usepackage{float}
%% pentru tabele pe mai multe pagini
\usepackage{longtable}	
%% pentru tabele cu celule cu mai multe linii
\usepackage{multirow}			
%% pentru ecuatii
\usepackage{amsmath}
%% pentru definitii, teoreme
\usepackage{amsthm}
%% pentru algoritmi, pachetul algorithm este cel mai basic
\usepackage[chapter]{algorithm}
\usepackage[noend]{algpseudocode}
% daca pachetul algorithm nu este satisfacator pentru ceea ce aveti de scris, mai exista pachetele:
% algorithmic, algorithmicx, algpseudocode, algorithm2e
% https://tex.stackexchange.com/questions/229355/algorithm-algorithmic-algorithmicx-algorithm2e-algpseudocode-confused
%% formatari listing
\usepackage{listings}
% https://www.overleaf.com/learn/latex/Code_listing
%% pachetul minted formateaza/coloreaza sintaxa codului sursa, spre deosebire de pachetul listings care doar face bold pe sintaxa
\usepackage[newfloat]{minted}
% https://www.overleaf.com/learn/latex/Code_Highlighting_with_minted


%% setup dimensiune pagina, dimensiune header/footer
%% setup dimensiune 16pt
%% valorile generice, pentru documentul in ansamblu
%% in mod uzual, nu ar trebui modificate

%% dimensiune pagina text, header/footer
\geometry{
    a4paper,
    left=2.5cm,
    right=2cm,
    top=2cm,
    bottom=2cm,
    headheight=0.55cm, %% daca am fi mers strict pe 0.5cm, am fi avut warning-uri
    footskip=0.55cm,
}

%% spatiere paragrafe si indentari
\linespread{1.0}                % spatierea intre randuri de 1 rand
\setlength{\parskip}{0pt}       % spatierea intre paragrafe
\setlength{\parindent}{1.27cm}  % identare paragraf de 1.27 cm

%% dimensiune font 16pt
\newcommand{\almostLarge}{\fontsize{16pt}{20pt}\selectfont}
%% date caracteristice universitatii/facultatii.
%% in acest fisier nu trebuie aduse modificari, cu exceptia tipului de teza si a anului in care este sustinuta teza
%% comune: coperta externa si interna
\newcommand{\university}    {UNIVERSITATEA TEHNICĂ „Gheorghe Asachi” din IAȘI}
\newcommand{\faculty}       {FACULTATEA DE AUTOMATICĂ ȘI CALCULATOARE}
\newcommand{\studyfieldlbl} {DOMENIUL: }
\newcommand{\studyfield}    {Calculatoare și Tehnologia Informației}
\newcommand{\studyproglbl}  {SPECIALIZAREA: }
\newcommand{\studyprog}     {Tehnologia Informației}
\newcommand{\promotion}     {2022}
\newcommand{\location}      {Iași}

%% absolventii de master vor inlocui "diploma" cu "disertatie"
\newcommand{\thesistype}    {Lucrare de diplomă}
%\newcommand{\thesistype}    {Lucrare de disertație}
%% datele specifice: nume, prenume student; titlu lucrare; nume, prenume, grad didactic coordonator
%% PENTRU MODIFICARE SE EDITEAZA FISIERUL macros/student.tex
%% date student si lucrare
\newcommand{\thesistitle}   {Analiză comparativă a metodelor clasice și cuantice
de generare a numerelor aleatoare. Aplicație.}    %<---------
\newcommand{\authorlast}    {Stoian}         %<---------
\newcommand{\authorfirst}   {Alin-Bogdan}
\newcommand{\authornamefl}  {\authorfirst \space \authorlast} % first name last
\newcommand{\authornamelf}  {\authorlast \space \authorfirst} % last name first

%% titlul academic si numele coordonatorului stiintific
\newcommand{\coordinator}   {Ș.l. dr. ing. Petrilă Iosif-Iulian}
%% pentru numele complet si grad didactic/titlu academic, consultati https://ac.tuiasi.ro/despre/personal/cadre-didactice-ale-departamentului-de-calculatoare/



\begin{document}

%% definitii culori
\definecolor{albastru-ac}{HTML}{004586}

\titleformat
    {\chapter}
    [hang]
    {\color{albastru-ac}\bfseries\large}
    {}
    {0pt}
    {}
    []
    
\titlespacing
    {\chapter}%
    {1.27cm}
    {0.42cm}
    {0.21cm}
    [0pt]

\titleformat
    {\section}
    [hang]
    {\color{albastru-ac}\bfseries\large}
    {}
    {0pt}
    {}
    []
    
\titlespacing
    {\section}%
    {1.27cm}
    {0.42cm}
    {0.21cm}
    [0pt]

\begin{center}
    \large
    {\textbf{\thesistype} -- \textbf{Raport nr. 2}}
    
    \vspace{0.5cm}
    
    \normalsize
    \textbf{\thesistitle}
\end{center}

\textit{Student}: \textbf{\authornamefl}

\textit{Coordonator științific}: \textbf{\coordinator}

\vspace{0.5cm}

Al doilea raport trebuie să aibă minim 5-6 pagini și să conțină:

\section*{Proiectarea hardware/software a aplicației}

La data la care ar trebui predat acest raport, aplicația proiectului de licență ar trebui să fie în stadiul în care proiectarea este finalizată, implementarea se află într-un stadiu avansat și ar trebui să fi fost realizate primele experimente și să fi obținut un set primar de rezultate experimentale.

În acest capitol va fi descrisă, pe scurt, arhitectura aplicației, modele/diagrame (elemente specifice Capitolului 2 din documentația lucrării de licență), precum și modul în care au ajuns să fie implementate (elemente specifice Capitolului 3).

În acest capitol nu trebuie furnizate amănunte, având rolul descrierii soluției tehnice de ansamblu.

\section*{Rezultate intermediare obținute}

Primele rezultate ar trebui să confirme soluția aleasă și să se încadreze, cu o marjă, în rezultatele așteptate descrise în primul raport intermediar.

Dacă sunt obținute anomalii, acestea ar trebui să aibă o explicație/justificare care ar trebui introdusă în acest capitol, eventual soluții de rafinare a rezultatelor care ar putea fi aplicate în timpul rămas până la predarea lucrării.

\section*{Dificultății/provocări întâmpinate și soluții de rezolvare}

În această secțiune pot fi incluse elementele corespunzătoare din capitolul al treilea al tezei.

Acest capitol nu trebuie să includă dificultăți din categoria „nu am găsit pe net soluția”, „mi-a fost greu să mă adaptez cu limbajul de programare ales” sau „resurse hardware insuficiente”.

Acest capitol ar putea fi organizat și sub forma Problemă întâmpinată / Soluție aleasă.
Ideal ar fi să fi identificat mai multe soluții posibile și ați ales, justificat, una dintre acestea pe baza unor argumente care țin de specificul temei alese, natura datelor prelucrate etc.

% *************** referinte bibliografice ***************
\nopagebreak
\bibliographystyle{IEEEtran}
\bibliography{bibliografie/bibliografie}

\end{document}