%% creez o comanda nou pentru capitolele ne-numerotate pentru a le putea prelua in header si a le putea adauga la cuprins;
%% in loc de \chapter*{....} va fi \silentchapter{....}
%% sursa de inspiratie pentru \silentchapter{....}: 
%% https://tex.stackexchange.com/questions/89914/chapter-name-in-the-header-with-chapter
\newcommand{\silentchapter}[1]{
    \chapter*{#1}
    \markboth{#1}{#1}
    \addcontentsline{toc}{chapter}{#1}
}

%% notele de subsol vor fi numerotate in continuu pe parcursul tezei
%% fara a mai fi resetate la fieare capitol
\counterwithout{footnote}{chapter}

%% setarile corespunzatoare antetului/subsolului unei pagini de continut
\pagestyle{fancy}
\fancyhf{}
%% pagina impara
%%      antet: titlul capitolului, aliniat in dreapta
%%      subsol: numarul paginii, aliniat in dreapta
\fancyhead[RO]{\nouppercase \leftmark}
\fancyfoot[RO]{\thepage}
%% pagina para
%%      antet: numele autorului, aliniat in stanga
%%      subsol: numarul paginii, aliniat in stanga
\fancyhead[LE]{\authornamefl}
\fancyfoot[LE]{\thepage}

%% setarile corespunzatoare antetului/subsolului unei pagini de inceput de capitol
%% se defineste prin \fancypagestyle{plain}{...}
\fancypagestyle{plain}{
    \fancyhf{}
    \fancyhead[RO]{\nouppercase \leftmark}
    \fancyfoot[RO]{\thepage}
}

%% definitii culori
\definecolor{albastru-ac}{HTML}{004586}

%% Titlu capitol
%% setari format
\titleformat
    {\chapter}%
    [hang]% titlul pe acelasi rand cu eticheta de capitol
    % bold, dimensiune 14pt (pentru normal 12pt), culoare
    {\color{albastru-ac}\bfseries\large}
    {Capitolul \thechapter.\ }% eticheta capitol
    {0pt}% separator
    {}% before-code -- nu setez nimic
    []% after-code -- nu setez nimic
%% setari spatiere
\titlespacing
    {\chapter}%
    {1.27cm}% aliniere la stanga
    {0.42cm}% spatiu inainte de titlul capitolului
    {0.42cm}% spatiu dupa titlul capitolului
    [0pt]% spatiu fata e marginea din dreapta

%% Subcapitol #.# (\section)
%% modul de lucru este acelasi, nu sunt reluate comentariile
\titleformat
    {\section}
    [hang]
    {\color{albastru-ac}\bfseries\itshape}
    {\thesection.\ }
    {0pt}
    {}
    []
\titlespacing
    {\section}%
    {1.27cm}
    {0.42cm}
    {0.21cm}
    [0pt]

%% Subsubcapitol #.#.# (\subsection)
%% modul de lucru este acelasi, nu sunt reluate comentariile
\titleformat
    {\subsection}
    [hang]
    {\color{albastru-ac}\itshape}
    {\thesubsection.\ }
    {0pt}
    {}
    []
\titlespacing
    {\subsection}%
    {1.27cm}
    {0.42cm}
    {0.21cm}
    [0pt]

%% Subsubsubcapitol #.#.#.# (\subsubsection)
%% modul de lucru este acelasi, nu sunt reluate comentariile
\titleformat
    {\subsubsection}
    [hang]
    {\color{albastru-ac}}
    {\thesubsubsection.\ }
    {0pt}
    {}
    []
\titlespacing
    {\subsubsection}%
    {1.27cm}
    {0.42cm}
    {0.21cm}
    [0pt]

%% configurari caption figura
%% - separator .
%% - dimensiune font 11pt (\small)
%% - culoare albastru (similar capitol)
%% - titlul apare centrat sub figura
\captionsetup[figure]{
    labelsep=period,
    font={small, color=albastru-ac},
    justification=centering,
    position=below
}
%% - numerotarea figurilor se face relativ la capitol (George Vieriu style)
\renewcommand{\thefigure}{\arabic{chapter}.\arabic{figure}}

%% configurari caption tabel
%% - separator .
%% - dimensiune font 11pt (\small)
%% - culoare albastru (similar capitol)
%% - titlul apare centrat deasupra figura
%% - namele intrarii de tabel
\captionsetup[table]{
    labelsep=period,
    font={small, color=albastru-ac},
    justification=centering,
    position=above,
    name=Tabelul
}
%% - numerotarea figurilor se face relativ la capitol (George Vieriu style)
\renewcommand{\thetable}{\arabic{chapter}.\arabic{table}}

%% - numerotarea ecuatiilor se face relativ la capitol (non George Vieriu style)
%% (este modul implicit de lucru)
%% - referirea unei ecuatii se face cu comanda \eqref{label:ecuatie}
%% - numerotarea definitiilor si teoremelor nu este precizata in template-ul lui George Vieriu
%% (din nou, am lasat modul implicit de lucru, si anume, relativ la capitol)
\theoremstyle{definition}
\newtheorem{definition}{Definiția}[chapter]
%% - teoreme
\theoremstyle{plain}
\newtheorem{theorem}{Teorema}[chapter]

%% configurari algoritmi
%% - denumire in romana
\floatname{algorithm}{Algoritmul}
%% - este definita o pseudo-instructiunie pentru "eticheta" de pas in proceduri
\makeatletter
\def\BState{\State\hskip-\ALG@thistlm}
\makeatother

%% configurari listing-uri de cod
%% - pachet minted
%% in loc de \begin{listing} {...} \end{listing} va fi folosit noul environment code: \begin{code} {...} \end{code}
%% ceea ce permite ca \begin{minted} {...} \end{minted} sa fie in interiorul \begin{code} {...} \end{code}
\newenvironment{code}{
    \captionsetup{
        type=listing,
        font={small, color=albastru-ac},
        labelsep=period,
        skip=-0.5em, % mut caption-ul putin mai aproape de cod
    }
}{\vspace{1.5em}}

%% - font pentru numarul liniei din cod
\renewcommand{\theFancyVerbLine}{
    \fontfamily{pcr}\small \arabic{FancyVerbLine}
}

%% setarea globala a optiunilor pentru minted, pentru a nu fi scrise de fiecare data dupa \begin{minted}
\setminted{
    frame=none,
    framesep=1mm,
    linenos=true,
    xleftmargin=1em,
    tabsize=2,
    fontfamily=courier,
    autogobble,
    fontsize=\small,
    escapeinside=|| %pentru a putea ava cod latex inside minted (referi linii de cod)
}

%% afisare DOI ca link in bibliografie
% - articolele de jurnal/conferinta pot avea un identificator de tip DOI (Digital Object Identifier)
% - in functie de indexarea articolului, aceasta informatie poate fi:
%       - doi.org - in mod uzual pentru jurnale/conferinte indexate ISI
%       - acm.org - pentru jurnale/conferinte indexate ACM
% - in bibliografie, DOI-ul se trece optional in campul "note", 
%   folosind una dintre cele doua comenzi definite mai jos
\newcommand*{\doi}[1]{DOI: \href{https://doi.org/\detokenize{#1}}{\detokenize{#1}}}
\newcommand*{\doiacm}[1]{DOI: \href{https://dl.acm.org/doi/\detokenize{#1}}{\detokenize{#1}}} 

