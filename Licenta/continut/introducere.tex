\silentchapter{Introducere}
\iffalse
Introducerea va avea 2–3 pagini care vor conține motivația alegerii temei, relevanța și contextul temei alese, obiectivele generale ale lucrării, metodologia și instrumentele utilizate și o scurtă descriere a structurii lucrării (titlul capitolelor, o scurtă descriere și legătura dintre acestea).

În acest capitol nu se introduc figuri, table sau listing-uri de cod. Pot fi în schimb referite!
\fi
\renewcommand{\thesection}{\Roman{section}}
\renewcommand{\thesubsection}{\thesection.\Roman{subsection}}
\section{Motivația alegerii temei}
\subsection{Numerele aleatoare}
Numerele aleatoare, aleatoritatea și generarea de numere aleatoare au aplicații nelimitate într-o mare mulțime de domenii diferite, precum în statistică, criptologie, jocuri, jocuri de noroc, matematică și chiar până și în artă.
Pentru câteva exemple de utilizare a numerelor aleatoare și a metodelor de generare a numerelor aleatoare, aflăm de la Aristotel în "Politica" sa \cite{book:aristotle:2015} că democrația din Grecia antică se baza deseori pe alegeri aleatoare\footnote{"... it is thought to be democratic for the offices to be assigned by lot, for them to be elected is oligarchic." [Aristotle, Politics 4.1294b]}, iar alegerile bazate pe vot puteau fi considerate chiar "oligarhice" în natură. Așadar, numerele aleatoare joacă un rol destul de important în dezvoltarea societății umane, încă de pe vremea celor mai timpurii civilizații!

Până și în ziua de astăzi, unele sisteme politice se bazează pe alegere aleatoare pentru sistemele lor juridice - de exemplu, în sistemul legal anglo-saxon, judecătorii din cadrul unui tribunal sunt deseori aleși aleator pentru a asigura imparțialitate, numerele aleatoare jucând astfel un rol în sistemul legislativ!

Pe lângă aplicațiile din politică, aleatoritatea joacă un rol important până și în sistemele religioase - alegerea unui nou Papă catolic fiind, bazată pe culoarea fumului care rezultă în urma arderii unor bilete de vot, pentru a oferi un exemplu din mai multe.

Aplicațiile antice sunt utile pentru a arăta faptul că numerele aleatoare joacă un rol în multiple aspecte ale societății umane din totdeauna, dar probabil că cea mai importantă aplicație a numerelor aleatoare în ziua de astăzi este in contextul criptologiei. O mare parte din metodele de securizare a sistemelor de comunicație în ziua de astăzi sunt bazate pe metode de generare a numerelor aleatoare (de exemplu, în cazul metodelor de autentificare sau în cazul metodelor de asigurare a confidențialității schimbului de date între mai multe sisteme de calcul). A fost dovedit matematic de către Shannon \cite{art:shannon:secrecy:1949} că orice cifru de criptare care asigură proprietatea de "discreție completă" \textit{[en. "perfect secrecy"]} \textbf{trebuie} să îndeplinească aceleași cerințe ca un sistem OTP (One-time pad), care este o tehnică de criptare bazată pe schimbul de o cheie, generată aleator, care nu poate fi mai scurtă decât mesajul trimis, între cele două sisteme de calcul aflate în schimb de mesaje.

Așadar, numerele aleatoare sunt extrem de importante pentru o mare parte din metodele de comunicare moderne (comerț online, sisteme tranzacționale, bancare, naționale, ș.a.m.d), pentru că majoritatea metodelor de creare de securitate sunt bazate, fundamental, pe o componentă aleatoare.

Totuși, dacă "importanța" unui fenomen nu se măsoară în impactul asupra societății, ci în cantitatea de bani generată de existența (sau inexistența) sa, atunci numerele aleatoare, fundamentale pentru industria jocurilor de noroc, sunt cu siguranță unul dintre cele mai importante concepte în existență. În momentul scrierii acestei lucrări, industria globală a cazinourilor și a jocurilor de noroc prezintă o piață în valoare de aproximativ 231 de miliarde de dolari \cite{misc:web:statista:gambling}, cu mult mai mult decât industria fotbalului, de exemplu, care are o valoare de piață de aproximativ 1.9 miliarde de dolari în 2019.

Aș putea să continui aproape la nesfârșit cu aplicațiile numerelor aleatoare într-o cantitate enormă de domenii, atât în trecut, cât și în prezent, dar ideea principală a lucrării de față este faptul de a implementa, a studia și a compara mai multe metode de generare de numere aleatoare, domeniu de studiu important din toate privințele.

\pagebreak

\section{Obiectivele lucrării}



\renewcommand{\thesection}{\arabic{section}}
\renewcommand{\thesubsection}{\thesection.\arabic{subsection}}




