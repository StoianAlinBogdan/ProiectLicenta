\newpage
\begin{center}
    \LARGE
    \textbf{\thesistitle}
    
    \vspace{0.5cm}
    
    \authornamefl
    
    \vspace{1cm}
    
    \textbf{Rezumat}
    
    \vspace{1cm}
\end{center}


Lucrarea de față își propune topica principală a analizei punctelor tari și a celor slabe a unor eșantioane reprezentative din toate categoriile de generatoare de numere aleatorii, adică propune compararea perfomanțelor și a calității numerelor generate pentru două generatoare de numere pseudoaleatorii (PRNG), un generator de numere aleatorii hardware (HRNG) și mai multe circuite cuantice care duc la generarea de numere aleatorii pe o distribuție uniformă (QRNG).

În acest sens, se prezintă conceptele teoretice ale funcționării QRNG-urilor, formulele de calcul a PRNG-urilor analizate, dar și concepte generale legate de funcționarea generatoarelor de numere aleatorii.

Pe lângă acestea, se prezintă aspecte legate de testele statistice și măsurătorile implementate pentru determinarea calității numerelor aleatorii.

Ulterior, se prezintă cerințele execuției proiectului și bibliotecile / framework-urile utilizate în toate aspectele lucrării, principale printre acestea fiind Qiskit pentru lucrul cu circuite cuantice și PySimpleGUI pentru aplicația cu interfață utilizator.

Apoi, se prezintă pe scurt implementarea PRNG-urilor, a QRNG-urilor și a testelor statistice descrise în primul capitol, apoi și a aplicației cu interfață utilizator. 

În cele din urmă, se prezintă măsurătorile de performanță a generatoarelor, printre care timpii de execuție, dar și parametrii statistici descriși anterior, pe baza cărora se trag niște concluzii. În acest sens, se determină că, în aproximativ toate aspectele în afara aleatorității globale (global randomness), PRNG-urile sunt net superioare.