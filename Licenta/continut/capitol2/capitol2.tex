\chapter{Proiectarea aplicației}
\label{cap:cap2}

(10–20 pagini)

\begin{itemize}
    \item se analizează platforma hardware pe care va fi executată respectiva aplicație și se analizează care abordare în implementare ar fi mai bună pentru respectiva structură
    \item se stabilesc modulele generale ale aplicației și interacțiunile dintre ele;
    \item se analizează avantaje și dezavantajele metodei alese;
    \item se indică limitele în care metoda va funcționa. 
\end{itemize}

\textit{Componente software:}
\begin{itemize}
    \item proiectarea propriu zisă (diagrame ER pentru baze de date, UML pentru proiectele care necesită diverse paradigme complexe și lucru cu clase – orientat obiect, scheme logice pentru cei care dezvoltă în limbaje structurate etc);
    \item se stabilește tehnologia aleasă pentru implementare și se justifică alegerea făcută;
    \item se descriu succint numai clasele dezvoltate și implementate de absolvent cu trimitere la pagina din anexă unde se află codul complet;
\end{itemize}

Figurile mai complexe, care nu se văd bine în format A4/portrait, pot fi incluse în anexe și referite normal (vezi Figura \ref{fig:AT} din Anexa \ref{anexa3:func_xyz}).

\vspace{1em}

\textit{Componente hardware:}
\begin{itemize}
    \item stabilirea componentelor hardware necesare. Exemplu: etaje analogice, display-uri, dispozitive I/O, periferice USB, etc.;
    \item analiza performanțelor și descrierea perifericelor procesorului/microcontrolerului folosit
    \item realizarea schemei bloc care să reflecte interconectarea componentelor principale;
    \item simularea funcționării componentelor hardware (OrCAD, Proteus, simulatoare HDL);
    \item proiectarea cablajului imprimat (Altium Designer, Eagle).
\end{itemize}

\textcolor{gray}{\lipsum} (Figura \ref{fig:BB8})

\begin{figure}[H]
    \centering
    \includegraphics[width=0.3\textwidth]{continut/capitol2/figuri/BB8.png}
    \caption{BB8\protect\footnotemark}
    \label{fig:BB8}
\end{figure}
\footnotetext{imagine preluată de pe un site web care nu „merită” trecut la bibliografie \url{https://www.pngitem.com/}}

\textcolor{gray}{\lipsum}

Ecuația \eqref{eq:sample} este un exemplu foarte simplu de formulă matematică. Ecuațiile vor fi referite prin comanda \LaTeX\ \verb|\eqref{}| din pachetul \verb|amsmath|.

\begin{equation}
    \label{eq:sample}
    e^{\pi i} + 1 = 0
\end{equation}

\begin{equation}
    \label{eq:multiline_sample}
    \begin{split}
        A & = \frac{\pi r^2}{2} \\
         & = \frac{1}{2} \pi r^2
    \end{split}
\end{equation}

În ecuația \eqref{eq:multiline_sample} avem un prim exemplu multiline.

Ecuațiile, definițiile și teoremele se referă în textul lucrării de licență/disertație prefixate: ecuația \eqref{eq:sample}, definiția \ref{def:diploma} sau teorema \ref{thm:equal}, iar aici este o pură întâmplare că au același indice, fiind vorba despre prima ecuație, prima definiție și prima teoremă din Capitolul \ref{cap:cap2}.

În continuare, definiția \ref{def:diploma} introduce conceptul de „Lucrare de diplomă”. Titlul efectiv este opțional și îl regăsiți în \LaTeX\ între [].

\begin{definition}[Lucrare de diplomă]
    \label{def:diploma}
    Lucrarea de diplomă face dovada nivelului și calității pregătirii profesionale, teoretice și aplicative a absolventului, iar prin modul în care aceasta este prezentată (susținută public în fața unei comisii de examinare) ea evidențiază calitățile științifice și inginerești definitorii ale absolventului.
\end{definition}

Numerotarea teoremelor se va face similar ecuațiilor și definițiilor de mai sus. Vezi teorema \ref{thm:equal}.

\begin{theorem}[Teorema egalității]
    \label{thm:equal}
    Dacă $a=b$, atunci $b=a$.
\end{theorem}

\begin{proof}
Ne bazăm pe proprietatea de reflexivitate.
\end{proof}

\textcolor{gray}{\lipsum}

În acest capitol ar trebui să se regăsească un algoritm (dacă este cazul) și el va fi referit ca Algoritmul \ref{alg:acceptance}.

\begin{algorithm}[ht]
    \caption{Pseudo code for reviewing process}\label{alg:acceptance}
    \begin{algorithmic}[1]
        \Procedure{Acceptance}{paper, committee}
            \State $\textit{reviewers} \gets \text{randomly select } \{(i,j) \mid i,j \in  \{1, \cdots, \#\{\textit{committee}\}\}, i \neq j \}$
            \State $\textit{reviews} \gets 2 \times 1 \text{ array of } \textit{False}$
            \BState \emph{reviewing round}:
            \For {$i \in \textit{reviewers }$}
                \If {$paper \text{ reads well for } \textit{committee}(i) $}
                    \State $\textit{review}(i) \gets \textit{True}$.
                \Else \text{ do nothing}
                \EndIf
            \EndFor
            \BState \emph{evaluation round}:
            \If {$\textit{reviews}(i) \text{ is } \textit{False} \text{ for all } i \in \textit{reviewers}$}
                \State \textbf{goto} \emph{hell}.
            \ElsIf {$\textit{reviews}(i) \text{ is } \textit{True} \text{ for all } i \in \textit{reviewers}$}
                \State \textbf{goto} \emph{Licența 2022}
            \Else \textbf{ goto} \emph{revision round}
            \EndIf
            \BState \emph{revision round}:
            \State $\textit{reviser} \gets \text{randomly select } \{i \mid i \in  \{1, \cdots, \#\{\textit{committee}\}\}, i \not\in \textit{reviewers} \}$
            \If {$paper \text{ reads well for } \textit{committee(reviser)} $}
                \State \textbf{ goto} \emph{Licența 2022}
            \ElsIf{\textit{committee(reviser)} \text{ has no opinion and likes inifinite loop}}
                \State \textbf{ goto} \emph{reviewing round}
            \Else \textbf{ goto} \emph{hell}
            \EndIf
            \BState \emph{hell} 
            \State \text{Try again in summer 2023!}
            \BState \emph{Licența 2022 (some people also call it hell)}:
            \State \text{See you at Beer Zone!}
        \EndProcedure
    \end{algorithmic}
\end{algorithm}

\textcolor{gray}{\lipsum}