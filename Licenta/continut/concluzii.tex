\silentchapter{Concluzii}

Concluziile lucrării (1–2 pagini) în care se regăsesc cele mai importante concluzii din lucrare, care pornind de la obiectivele propuse în introducerea lucrării vor conține opinia personală a autorului privind rezultatele obținute în lucrare, o comparație a acestora cu alte proiecte/produse similare precum și posibile direcții de dezvoltare. 

\begin{itemize}
    \item Gradul în care s-a realizat tema propusă (motivarea eventualelor obiective modificate);
    \item Evidențierea concisă a contribuțiilor/soluțiilor personale (dacă este cazul);
    \item Comparație cu alte proiecte similare;
    \item Posibile direcții de dezvoltare.
\end{itemize}

În acest capitol nu se introduc figuri, table sau listing-uri de cod. Pot fi în schimb referite!

Concluziile finale au un rol semnificativ în cadrul lucrării de diplomă, care trebuie pus în evidență în special în momentul susținerii acesteia în fața comisiei. Aici trebuie prezentate, în câteva pagini, cele mai importante rezultate ale muncii depuse de candidat și potențiale direcții de dezvoltare ulterioară a temei abordate. Acest capitol oferă o sinteză a realizărilor obținute, precum și avantajele și neajunsurile soluțiilor alese. Concluziile nu trebuie să aducă rezultate sau interpretări noi, ci doar să le evidențieze pe cele din conținutul lucrării.

\section*{Direcții viitoare de dezvoltare}

\textcolor{gray}{\lipsum[1-3]}

\section*{Lecții învățate pe parcusul dezvoltării lucrării de diplomă}

\textcolor{gray}{\lipsum[2-4]}