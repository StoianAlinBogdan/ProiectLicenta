\iffalse
\chapter{Implementarea aplicației}
\label{cap:cap3}

(10–15 pagini)

\begin{itemize}
    \item Descrierea generală a implementării;
    \item Probleme speciale/dificultăți întâmpinate și modalități de rezolvare;
    \item Idei originale, soluții noi;
    \item Se prezintă pe scurt funcționarea sistemului (câteva capturi de ecran în punctele esențiale); nu se insistă deosebit deoarece există prezentare practică
    \item Comunicarea cu alte sisteme și salvarea/stocarea informațiilor;
    \item Interfața cu utilizatorul; 
    \item Se realizează calibrarea hardware și eventual software și se dau detalii despre maniera în care a fost efectuată. 
\end{itemize}

Dacă în Capitolul \ref{cap:cap2} este descrisă arhitectura aplicației la nivel conceptual (cu scheme logice, diagrame UML, structuri de clase, diagrame ER etc.), Capitolul \ref{cap:cap2} descrie implementarea soluției propuse. Această descriere trebuie să urmeze modulele prezentate în capitolul anterior, cu o eventuală referire a acestora. Se pot introduce fragmente semnificative de cod specifice fiecărui modul în parte.

În acest capitol se descrie implementarea practică a proiectului. În mod uzual, puteți insera cod de dimensiuni mici, care este relevant și care se impune a fi comentat în detaliu. Astfel, Listing-ul \ref{code:c_mpi} prezintă un supergreu pas de inițializare a unui program MPI pentru a obține cinstit punctul din oficiu, în care se poate face referire la linia \ref{line:if} pentru a descrie ce face \verb|if|-u' di acolo.

\begin{code}
    \begin{minted}{c}
        #include <stdio.h>
        #include "mpi.h"
        
        int main( int argc, char **argv ) {
        	char message[20];
        	int myrank;	
        		
        	MPI_Status status;
        
        	MPI_Init( &argc, &argv );
        	MPI_Comm_rank( MPI_COMM_WORLD, &myrank );
        	
        	if (myrank == 0) { /* codul pentru procesul 0 */|\label{line:if}|
        		strcpy(message,"Hello");
        		MPI_Send(message, strlen(message), MPI_CHAR, 
        			1, 99, MPI_COMM_WORLD);
        	} else { /* codul pentru procesul 1 */
        		MPI_Recv(message, 20, MPI_CHAR, 
        			0, 99, MPI_COMM_WORLD, &status);
        		printf("received :%s:\n", message);
        	}
        	
        	MPI_Finalize();
        	return 0;
        }
    \end{minted}
    \caption{Supercod MPI în C} 
    \label{code:c_mpi}
\end{code}

Codul complet relevant se va include în anexe (vezi Anexele \ref{anexa6:listing_python}, \ref{anexa7:listing_kotlin} și \ref{anexa8:listing_xml}). De exemplu, Listing-ul \ref{code:xml_pom} (Anexa \ref{anexa8:listing_xml}) prezintă fișierul de configurare \textit{maven} pentru un serviciu SOAP minimal.

\textcolor{gray}{\lipsum}

\textcolor{gray}{\lipsum}
\fi
